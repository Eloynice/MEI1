\documentclass[runningheads]{llncs}

\usepackage{graphicx}
\usepackage{float}
\usepackage{biblatex}
\addbibresource{references.bib}
\graphicspath{ {./images/} }
\raggedbottom

\begin{document}

\title{Applied Artificial Intelligence}

\author{Eduardo Eloy}

\institute{Universidade de Évora, Portugal}


\maketitle              

\begin{abstract}
This paper provides an implementation of a constraint programming solution to the Costas Array Problem that through a Difference Triangle built with the Costas Array ensures that no two lines in the array have the same length and slope.
Constraint programming being a powerful tool for finding and optimising solutions to these kinds of combinatorial problems makes it, alongside MiniZinc, a good choice to solve this problem widely used in military applications.


\keywords{Constraint \and Costas \and Difference.}
\end{abstract}

\section{The Problem}
\paragraph{}

Out of the 4 available options I chose the Costas Array problem\cite{costasp}. These types of arrays are named after John P. Costas\cite{costas}, who first wrote about them in 1965 and have since been used in military applications of radar and sonar systems to, for example, detect submarines.
The problem can be described as a N*N matrix of Unmarked (U) and Marked (M) elements where each row and column may only have one M element and no two lines formed between pairs of M elements may have the same length and the same slope, meaning every pair of of M elements must form a unique line segment.
Line segments may have the same length but different slopes or the same slopes but different lengths.


\section{State of The Art}
\paragraph{}

The Costas array problem has been solved and as of 06/01/2021 the On-Line Encyclopedia of Integer Sequences has gathered solutions to the problem up to problem size 29\cite{int} although there are known solutions with a larger size.
There are some popular constructions of Costas arrays like the Welch–Costas construction, the Lempel construction and the Golomb construction, however I won't be using any of these constructions to solve or model the problem. Parallel computing has also been used in solutions of this problem

In January of 2012 an article, by Salvador Abreu Et Al\cite{spa}, was written proposing the solving of Costas arrays problem through a type of Constraint based Local search method called Adaptive search\cite{adaptive}, in doing so, it proposes a model of the Costas array as a permutation of N, similarly to the N-queens\cite{queens} model we are used to solving.
Because this gives e a familiar setup to solve the problem I will orient myself by the guidance of this article and attempt to implement this type of construction.


\section{The Tool}
\paragraph{}

The Minizinc\cite{minizinc} constraint modelling language is a powerful modelling tool that allows the use of several pre-defined constraints. The model and constraints are compiled into FlatZinc, a solver input language that is understood by a wide range of solvers like Gecode, Chuffed and Coin-BC. It as extensive documentation and can be installed along with a specialized IDE


\section{The Model}
\paragraph{}

As mentioned earlier, the matrix is represented through an integer array of N values, this array being a permutation of N and having N! possible permutations.
The value at each index of the array refers to the position of the M element, this makes it so we apply an implicit constraint so no two M elements can be at the same row (or column depending on how you see it).
Due to being a permutation, each value in the array must be different.
Now the interesting part about assuring that the permutation we reach is a Costas array is that it relies on a modification of a structure called a "Difference Triangle".
Normally with a regular difference triangle we start with an initial array of N integers and we calculate the difference between the values at position x and position x+1 until we reach the end of the array.
This leaves us with a new array of N-1 integers and we repeat this cycle until we get to a single integer.
For our purpose though, the algorithm that builds our triangle is a bit different.
For each array position and for each array position higher than that first position, we subtract the value in the higher position of the array by the value in the position obtained by subtracting the higher position and the first position.
By doing that we obtain N-1 arrays of 1..N-1 values, if for each array all values are different, then the initial array is a Costas array.


Putting this in more mathematic terms we have the following

Problem size is N

Array $\alpha$ has N values

Values must go from 1 to N and must be different: $$\forall v \hspace{0.1cm}  in \hspace{0.1cm} \alpha, v \in {1..N}$$ 
$$\forall i,j \in \alpha , i\ne j$$

To build the difference triangle T, which is 2 dimensional array: $$\forall i,j \in \alpha \hspace{0.1cm} where \hspace{0.1cm} i < j,  T[i,j] = \alpha[j] - \alpha[j-i]$$

Every array in T must have all different values: $$\forall i\in{1..N}, \forall j,k \in {1..N}, T[i,j] \ne T[i,k]$$

\section{The Results}
\subsection{Specifications}
\paragraph{}

These results were obtained by compiling and running the programs on my PC which has the following specifications: \hfill\break
Memory: 7,6 GiB\hfill\break
Processor: Intel® Celeron(R) N4100 CPU @ 1.10GHz × 4\hfill\break
Graphics: Mesa Intel® UHD Graphics 600 (GLK 2)\hfill\break
OS: Ubuntu 20.04.1 LTS 64-bit\hfill\break

\subsection{NaiveMethod}
\paragraph{}

\begin{table}[H]
\centering
\caption{Time to find the first solution, Naive method}\label{NAVF}
\begin{tabular}{|l|l|l|}
\hline
Size & Real & User\\
\hline
1 & 0m0,295s & 0m0,110s\\
2 & 0m0,286s & 0m0,155s\\
3 & 0m0,260s & 0m0,138s\\
4 & 0m0,260s & 0m0,143s\\
5 & 0m0,244s & 0m0,122s\\
6 & 0m0,294s & 0m0,143s\\
7 & 0m0,375s & 0m0,169s\\
8 & 0m0,267s & 0m0,168s\\
9 & 0m0,261s & 0m0,164s\\
10 & 0m0,257s & 0m0,162s\\
11 & 0m0,289s & 0m0,173s\\
12 & 0m0,342s & 0m0,181s\\
13 & 0m0,410s & 0m0,275s\\
14 & 0m2,083s & 0m1,878s\\
15 & 0m15,855s & 0m14,954s\\
16 & 0m58,398s & 0m55,523s\\
17 & 7m49,817s & 7m44,406s\\
18 & 0m48,189s & 0m47,642s\\
19 & 345m49,232s & 341m49,306s\\
\hline
\end{tabular}
\end{table}

\begin{table}[H]
\centering
\caption{Time to find all solutions, Naive method}\label{NAVA}
\begin{tabular}{|l|l|l|}
\hline
Size & Real & User\\
\hline
1 & 0m0,169s & 0m0,113s\\
2 & 0m0,173s & 0m0,091s\\
3 & 0m0,171s & 0m0,104s\\
4 & 0m0,287s & 0m0,155s\\
5 & 0m0,166s & 0m0,100s\\
6 & 0m0,235s & 0m0,156s\\
7 & 0m0,328s & 0m0,197s\\
8 & 0m0,437s & 0m0,352s\\
9 & 0m1,216s & 0m1,116s\\
10 & 0m5,196s & 0m4,854s\\
11 & 0m24,013s & 0m23,543s\\
12 & 2m18,867s & 2m9,116s\\
13 & 11m2,120s & 10m47,396s\\
14 & 53m58,181s & 53m34,183s\\
15 & 315m10,565s & 312m22,656s\\
\hline
\end{tabular}
\end{table}

\section{Improved method}
\subsection{The Improved Model}
\paragraph{}

The improved model is based on adding 3 redundant constraints to the naive model and modifying the search strategy.
A redundant constraint is a constraint that is already logically implied by the the existing model but adding them may improve the search time
Minizinc has a "redundant constraint" function to specify the nature of these constraints.

One of these redundant constraints ensure the already implied constraint that the values in the Costas Array higher than 0 but lower than or equal to N.

The other 2 enforce the same constraint that all values in the Costas Array are different (which is already implied through the global "all different" constraint), one of them does this by iterating through the Costas Array and saying that the value in each index must be different than the values in the indices after it.
The other one enforces an all different constraint on the Costas Array but through the Difference Triangle, we iterate through the already built 2 dimensional Difference Triangle, and anytime i is lower than j (i being the first iteration, through the rows, and j being the second iteration, through the columns) the value at Difference Triangle[i,j] must be different than 0, because if it is 0 that would mean two values in the Costas Array were equal since these values are obtained by computing CostasArray[j] - CostasArray[j-i].

The search strategy was changed from the default strategy to an "Input Order Indomain Min" strategy, meaning that when the solver is choosing which variable's value it is going to try and find and from its domain which value it is going to try (to see it doesn't break any constraints) it is going to do this from the order of the array, starting with the first index and going forward, and from the values in the domain it's going to choose the lowest value.

I arrived at this strategy by trying some of the other options, like "First Fail", where the solver goes from the variable with the smallest domain size to the next, and "Indomain Median", where the median value in the domain is chosen, and comparing the results to find the combination that performed better.

Another constraint I added was a "symmetry breaking constraint" so that the solver wouldn't propose solutions that are just "inversions" of other solutions. This is done by constraining a variable at a specific index to be lower or greater than the variable at its "inverse" index, for example the inverse of the first index is the last one. So by declaring the CostasArray[1] must be lower than CostasArray[N] we are get solutions like [1, 2, 4, 3] but not its inverse [3,4,2,1].

\subsection{Results}
\paragraph{}
\begin{table}[H]
\centering
\caption{Time to find the first solution, Improved method}\label{IAVF}
\begin{tabular}{|l|l|l|}
\hline
Size & Real & User\\
\hline
1 & 0m0,213s & 0m0,107s\\
2 & 0m0,348s & 0m0,134s\\
3 & 0m0,218s & 0m0,115s\\
4 & 0m0,227s & 0m0,148s\\
5 & 0m0,299s & 0m0,128s\\
6 & 0m0,286s & 0m0,169s\\
7 & 0m0,246s & 0m0,118s\\
8 & 0m0,240s & 0m0,153s\\
9 & 0m0,245s & 0m0,149s\\
10 & 0m0,256s & 0m0,150s\\
11 & 0m0,333s & 0m0,184s\\
12 & 0m0,249s & 0m0,164s\\
13 & 0m0,423s & 0m0,223s\\
14 & 0m1,095s & 0m0,972s\\
15 & 0m9,112s & 0m8,922s\\
16 & 0m34,591s & 0m33,417s\\
17 & 7m5,511s & 6m55,474s\\
18 & 0m33,119s & 0m29,748s\\
19 & 180m48,177s & 180m32,328s\\
20 & 172m22,697s & 171m15,115s\\
\hline
\end{tabular}
\end{table}

\begin{table}[H]
\centering
\caption{Time to find all solutions, Improved method}\label{IAVA}
\begin{tabular}{|l|l|l|}
\hline
Size & Real & User\\
\hline
1 & 0m0,250s & 0m0,137s\\
2 & 0m0,374s & 0m0,163s\\
3 & 0m0,514s & 0m0,158s\\
4 & 0m0,350s & 0m0,133s\\
5 & 0m0,409s & 0m0,147s\\
6 & 0m0,311s & 0m0,162s\\
7 & 0m0,524s & 0m0,219s\\
8 & 0m0,722s & 0m0,324s\\
9 & 0m1,131s & 0m0,600s\\
10 & 0m3,852s & 0m2,742s\\
11 & 0m17,382s & 0m13,716s\\
12 & 1m19,371s & 1m12,336s\\
13 & 7m15,399s & 6m38,845s\\
14 & 33m26,841s & 32m55,638s\\
15 & 140m56,205s & 139m33,558s\\
\hline
\end{tabular}
\end{table}

\section{Overall Assessment}
\paragraph{}

In the end both methods perform similarly, the only outlier being finding the first solution with the naive method with a size of 17, the improved method does find solutions faster and in the long run that results in also finding all solutions faster but finding solutions when the size is 19/20 begins to take tens of minutes or even hours, and finding all solutions when the size is 15 starts to take hours.

Although the improved method doesn't perform significantly better than the naive method it still is a light improvement and overall I'm satisfied with the result.


\printbibliography
\end{document}